% little trick to replace lib.tex by this
\renewcommand{\doctitle}[1]{
	\chapter{#1}
}
\renewcommand{\biblio}[1]{}
\doctitle{L'oscillateur contrôlé en tension (VCO)}

La figure 1 nous montre le schéma bloc du VCO.


Le VCO se compose des 3 blocs suivants:
\begin{itemize}
\item l'integrator controller (le contrôleur de l'intégrateur) qui se compose lui-même d'un switch et d'un summing (sommateur)
\item l'integrator (intégrateur)
\item le trigger de Schmitt (bascule à hystérèse)
\end{itemize}

Le signal d'entrée ($in$) est constant et vaut $\alpha$. Appelons $V_L$ la tension de basculement inférieure du trigger, $V_H$ la tension de basculement supérieure et $V_{CC}$ la tension d'alimentation du trigger.
Si la sortie du trigger de Schmitt ($control$) vaut $0$, la sortie du switch de l'integrator controller vaut $0$. Dès lors, la sortie du bloc integrator controller vaut $-\alpha$.  Après passage dans l'integrator, nous avons la droite $-\alpha t$. La sortie du trigger restera à $0$ tant que la sortie de l'integrator est supérieure à $V_L$. Lorsque la sortie de l'integrator a atteint $V_L$, le trigger bascule et sa tension de sortie devient $V_{CC}$. Le switch change d'état et sa sortie devient $\alpha$. La sortie de l'integrator controller devient donc $\alpha$. Après passage dans l'integrator, nous avons la droite $\alpha t$. La sortie du trigger restera à $V_{CC}$ tant que la sortie de l'integrator est inférieure à $V_H$. Lorsque la sortie de l'integrator a atteint $V_H$, le trigger bascule et sa tension de sortie devient $0$. Le switch rechange d'état et sa sortie devient $0$. Et nous pouvons recommencer la même boucle.

\end{document}